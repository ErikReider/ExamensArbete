\documentclass{article}
\usepackage[utf8]{inputenc}

\title{Unity game performance difference between Vulkan and WebGL in Linux}
\author{Erik Reider}
\date{January 2021}

\usepackage{natbib}
\usepackage{graphicx}

\begin{document}

\maketitle

\section{Background}
The Unity game engine (also referred to as Unity) is a game creation tool that is considered to be one of the most used game engines in the world. Unity can build games for Windows, Linux, Mac OS, Android, iOS, every modern console and your web browser. When compiling (building the project) towards Linux, the preferred graphics API (an acronym for Application Programming Interface) is Vulkan while the web browser uses WebGL2 (OpenGL ES 3.0 but will be referred to as WebGL). The graphics API makes it easier for the developers to render graphics more easily instead of writing code for every graphics card in existence. \par
OpenGL ES is a stripped down version of the regular more feature rich OpenGL while the differences between the latter and Vulkan is more stark. Vulkan is the successor to OpenGL which brings more asynchronous CPU optimizations and has less overhead but because of it being relatively new, it is not as widely adopted as OpenGL. Ergo, one of the reasons why WebGL uses OpenGL instead of Vulkan. One of the other reasons why WebGL doesn’t use Vulkan is because of the lack of native support for it on some platforms like Mac OS and iOS which uses OpenGL and Apple’s Metal API.


\section{Description}
Because WebGL has a tendency of being less resource efficient than Vulkan, there should be a real performance difference between the two, especially on lower end devices like older computers and consoles, phones, smart fridges etc, where the CPU performance isn’t the best. So focusing on CPU limited situations is an excellent way of benchmarking the differences. The benchmarks should be conducted on a low-end PC and a high-end PC to verify the performance differences. \par
To make the comparison fair, the high-end PC’s CPU will be underclocked to 2.0GHz to simulate a lower-end PC with the same hardware to eliminate any differences except for the CPU. While testing, the CPU usage, FPS and the GPU usage will be logged.


\section{Procedure}
Blender will be used to create the game scene within the Unity engine. The results will be graphed and displayed accordingly. The relevant courses are Programming 1, 2 and the digital creation course.

\section {Boundaries}
There will only be one benchmark to test with three reruns to verify the performance results.

\section {TimeTable}
\begin{center}
\begin{tabular}{ |c|c| } 
 \hline
    \textbf {Task} & \textbf {Time to spend} \\ [0.5ex] 
 \hline\hline
    Research CPU bound tests & 1-2 days \\ 
    \hline
    Implement said tests & 3 days \\ 
    \hline
    Test and increase “difficulty” if needed & 1 day \\ 
 \hline
\end{tabular}
\end{center}


\section {Method}
Each hardware setup will run the benchmark three times to eliminate any anomalies while testing. While running the benchmark, the FPS will be logged and averaged into separate files per run to ease the comparison between different results and to eliminate any inevitable mistakes. Before running the benchmarks, all non essential programs like Discord, Spotify, Steam will be killed to further eliminate any strange deviations between each run. To make the benchmark more CPU bound, all settings will be set to the lowest value which will increase the performance delta if there is one.

\begin{itemize}
    \item CPU: AMD Ryzen 7 5800X
    \item CPU performance profile: Performance
    \item Graphics driver: Mesa 21.1.0-devel (git-470d3a3640)
    \item Resolution: 1920x1080
    \item OS and kernel: Linux 5.10.7-tkg
\end{itemize}


\bibliographystyle{plain}
\bibliography{references}
\end{document}
